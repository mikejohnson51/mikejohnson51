%!TEX TS-program = xelatex
%!TEX encoding = UTF-8 Unicode
% Awesome CV LaTeX Template for CV/Resume
%
% This template has been downloaded from:
% https://github.com/posquit0/Awesome-CV
%
% Author:
% Claud D. Park <posquit0.bj@gmail.com>
% http://www.posquit0.com
%
%
% Adapted to be an Rmarkdown template by Mitchell O'Hara-Wild
% 23 November 2018
%
% Template license:
% CC BY-SA 4.0 (https://creativecommons.org/licenses/by-sa/4.0/)
%
%-------------------------------------------------------------------------------
% CONFIGURATIONS
%-------------------------------------------------------------------------------
% A4 paper size by default, use 'letterpaper' for US letter
\documentclass[11pt,a4paper,]{awesome-cv}

% Configure page margins with geometry
\usepackage{geometry}
\geometry{left=1.4cm, top=.8cm, right=1.4cm, bottom=1.8cm, footskip=.5cm}


% Specify the location of the included fonts
\fontdir[fonts/]

% Color for highlights
% Awesome Colors: awesome-emerald, awesome-skyblue, awesome-red, awesome-pink, awesome-orange
%                 awesome-nephritis, awesome-concrete, awesome-darknight

\definecolor{awesome}{HTML}{414141}

% Colors for text
% Uncomment if you would like to specify your own color
% \definecolor{darktext}{HTML}{414141}
% \definecolor{text}{HTML}{333333}
% \definecolor{graytext}{HTML}{5D5D5D}
% \definecolor{lighttext}{HTML}{999999}

% Set false if you don't want to highlight section with awesome color
\setbool{acvSectionColorHighlight}{true}

% If you would like to change the social information separator from a pipe (|) to something else
\renewcommand{\acvHeaderSocialSep}{\quad\textbar\quad}

\def\endfirstpage{\newpage}

%-------------------------------------------------------------------------------
%	PERSONAL INFORMATION
%	Comment any of the lines below if they are not required
%-------------------------------------------------------------------------------
% Available options: circle|rectangle,edge/noedge,left/right

\name{Dr.~J Michael}{Johnson}

\position{Geographer \textbar{} Data Scientist \textbar{} Water
Resources}
\address{Fort Collins, Colorado}

\email{\href{mailto:jjohnson@lynker.com}{\nolinkurl{jjohnson@lynker.com}}}
\homepage{mikejohnson51.github.io}
\googlescholar{MrXM9cgAAAAJ}
\github{mikejohnson51}

% \gitlab{gitlab-id}
% \stackoverflow{SO-id}{SO-name}
% \skype{skype-id}
% \reddit{reddit-id}

\quote{I am a \textcolor{cyan}{\textbf{geospatial data scientist}}
leading the
\textcolor{cyan}{\textbf{hydrofabric development for NOAA's Next Generation National Water Model}}
along with collaborative federal efforts to define a national suite of
hydroinformatic data products. I seek to bridge
\textcolor{cyan}{\textbf{data-intensive computational geography}} with
\textcolor{cyan}{\textbf{water resources research}} to design new data
products and develop open-source software to ease community access to
big geospatial data.}

\usepackage{booktabs}

\providecommand{\tightlist}{%
	\setlength{\itemsep}{0pt}\setlength{\parskip}{0pt}}

%------------------------------------------------------------------------------



% Pandoc CSL macros
\newlength{\cslhangindent}
\setlength{\cslhangindent}{1.5em}
\newlength{\csllabelwidth}
\setlength{\csllabelwidth}{2em}
\newenvironment{CSLReferences}[2] % #1 hanging-ident, #2 entry spacing
 {% don't indent paragraphs
  \setlength{\parindent}{0pt}
  % turn on hanging indent if param 1 is 1
  \ifodd #1 \everypar{\setlength{\hangindent}{\cslhangindent}}\ignorespaces\fi
  % set entry spacing
  \ifnum #2 > 0
  \setlength{\parskip}{#2\baselineskip}
  \fi
 }%
 {}
\usepackage{calc}
\newcommand{\CSLBlock}[1]{#1\hfill\break}
\newcommand{\CSLLeftMargin}[1]{\parbox[t]{\csllabelwidth}{\honortitlestyle{#1}}}
\newcommand{\CSLRightInline}[1]{\parbox[t]{\linewidth - \csllabelwidth}{\honordatestyle{#1}}}
\newcommand{\CSLIndent}[1]{\hspace{\cslhangindent}#1}

\begin{document}

% Print the header with above personal informations
% Give optional argument to change alignment(C: center, L: left, R: right)
\makecvheader

% Print the footer with 3 arguments(<left>, <center>, <right>)
% Leave any of these blank if they are not needed
% 2019-02-14 Chris Umphlett - add flexibility to the document name in footer, rather than have it be static Curriculum Vitae
\makecvfooter
  {October, 2023}
    {Dr.~J Michael Johnson~~~·~~~Curriculum Vitae}
  {\thepage}


%-------------------------------------------------------------------------------
%	CV/RESUME CONTENT
%	Each section is imported separately, open each file in turn to modify content
%------------------------------------------------------------------------------



\hypertarget{employment}{%
\section{Employment}\label{employment}}

\begin{cventries}
    \cventry{Chief Data Scientist/ Pod Lead}{Lynker}{Fort Collins, Colorado}{Sep 2023 - Present}{}\vspace{-4.0mm}
    \cventry{Water Resources Data Scientist}{}{}{Aug 2020 - Sep 2023}{\begin{cvitems}
\item Lead spatial data development for the NOAA NextGen Water Resource Modeling Framework
\item Contribute to local and state level consulting projects related to water resource managment and hazard mitigation
\item Recruit, retain, and mentor a strong and diverse group of data scientists
\end{cvitems}}
    \cventry{Hydrofabric Technical Director}{NOAA Office of Water Prediction}{Remote}{Sep 2022 - Present}{\begin{cvitems}
\item Develop, document, and publish foundational geospatial products to support version 4 of the National Water Model
\item Collaborate with the USGS to build a suite of tools and data products supporting the National Hydrologic Geospatial Fabric
\item Lead a team developing novel machine learning, geospatial, and cloud based solutions of more open and skilled science
\end{cvitems}}
    \cventry{Lead Hydrofabric Developer}{}{}{Aug 2020 - Sep 2022}{}\vspace{-4.0mm}
    \cventry{Lead Data Scientist (Independent Contractor)}{Urban Flooding Open Knowledge Network}{Remote}{Nov 2019 - Apr 2023}{\begin{cvitems}
\item Co-authored successful proposals to NSF and acted as an advocate for the team in the initial C-ACCEL program
\item Developed and designed a cost effective, cloud native, building level, flood forecasting system for the Continental United States.
\end{cvitems}}
    \cventry{Lecturer}{Geography Department}{UC Santa Barbara}{Summer 2020, 2021}{\begin{cvitems}
\item Designed and taught the first programming based GIS course for UC Santa Barbara in R.
\end{cvitems}}
    \cventry{Research Coordinator}{NOAA Office of Water Prediction}{Tuscaloosa, AL}{2016}{\begin{cvitems}
\item Led students towards the successful execution of projects related to the National Water Model
Research Fellow
\item Worked at the National Water Center in advancement of the National Water Model
\end{cvitems}}
    \cventry{Graduate Student}{Visiting Researcher}{}{}{\begin{cvitems}
\item \textbf{Institute for Environmental Studies. Vrije Universiteit}, Amsterdam, Netherlands: June - July 2019; January - March 2018
\item \textbf{Research Applications Laboratory. NCAR}, Boulder, Colorado: August - September 2018
\item \textbf{NOAA National Water Center}. Tuscaloosa, Alabama: Summers of 2016.2017
\end{cvitems}}
\end{cventries}

\hypertarget{education}{%
\section{Education}\label{education}}

\begin{cventries}
    \cventry{Santa Barbara, CA}{University of California, Santa Barbara}{PhD in Geography}{2021}{\begin{cvitems}
\item \textbf{Advisor}: Dr. Keith C. Clarke
\item \textbf{Committee}: Hugo Loaiciga, Kelly Caylor, David Blodgett
\item \textbf{Title}: The Role of Spatial Data Science in Continental Scale Hydrology: Twelve Case Studies in Data Models, Data Structures, Modeling, and Evaluation
\end{cvitems}}
    \cventry{San Luis Obispo, CA}{California Polytechnic State University}{BS in Anthropology \& Geography}{2010 - 2015}{\begin{cvitems}
\item Cum Laude
\item Outstanding Senior Award: College of Liberal Arts
\item \textbf{Minors}: \textbf{(1)} GIS for Agriculture \textbf{(2)} Water Science (Watershed Management) \textbf{(3)} Statistics \textbf{(4)} Economics \textbf{(5)} Environmental Studies
\end{cvitems}}
\end{cventries}
\newpage

\hypertarget{publications}{%
\section{Publications}\label{publications}}

\textcolor{cyan}{ \href{https://tinyurl.com/mike-google-scholar}{\faicon{google} Google Scholar: 389 citations; \faicon{user} 20 collaborators; \faicon{file} 19 papers} \newline
\textbf{h-index} 9; \textbf{i-index} 8}

\hypertarget{bibliography}{}
\leavevmode\vadjust pre{\hypertarget{ref-4}{}}%
Blodgett, D., \& \textbf{Johnson, J.} (2023). Hydrologic modeling and
river corridor applications of HY\_features concepts. \emph{OGC Public
Engineering Report}.

\leavevmode\vadjust pre{\hypertarget{ref-2}{}}%
Blodgett, D., \textbf{Johnson, J.}, \& Andy, B. (2023). Generating a
reference flow network with improved connectivity to support durable
data integration and reproducibility in the coterminous US.
\emph{Environmental Modelling \& Software}.

\leavevmode\vadjust pre{\hypertarget{ref-5}{}}%
Kohanpur, A., Saksena, S., Dey, S., \textbf{Johnson, J.}, Riasi, M.,
Yeghiazarian, L., \& \ldots. (2023). Urban flood modeling: Uncertainty
quantification and physics‐informed gaussian processes regression
forecasting. \emph{Water Resources Research}, \emph{59 (3),
e2022WR033939}.

\leavevmode\vadjust pre{\hypertarget{ref-3}{}}%
Montello, D., Davis, R., \textbf{Johnson, J.}, \& Chrastil, E. (2023).
The symmetry and asymmetry of pedestrian route choice. \emph{Journal of
Environmental Psychology}, \emph{102004}.

\leavevmode\vadjust pre{\hypertarget{ref-1}{}}%
Rad, A., Abatzoglou, J., Fleishman, E., Mockrin, M., Radeloff, V.,
Pourmohamad, Y., Cattau, M., \textbf{Johnson, J.}, Higuera, P., Nauslar,
N., \& Sadegh, M. (2023). Social vulnerability of the people exposed to
wildfires in US west coast states. \emph{Science Advances}, \emph{9
(38), eadh4615}.

\leavevmode\vadjust pre{\hypertarget{ref-6}{}}%
\textbf{Johnson, J.}, Narock, T., Singh-Mohudpur, J., Fils, D., Clarke,
K., Saksena, S., \& \ldots. (2022). Knowledge graphs to support
real-time flood impact evaluation. \emph{AI Magazine}, \emph{43 (1),
40-45}.

\leavevmode\vadjust pre{\hypertarget{ref-7}{}}%
\textbf{Johnson, J.}, \& Clarke, K. (2021). An area preserving method
for improved categorical raster resampling. \emph{Cartography and
Geographic Information Science}, \emph{48 (4), 292-304}.

\leavevmode\vadjust pre{\hypertarget{ref-10}{}}%
Blodgett, D., \textbf{Johnson, J.}, Sondheim, M., Wieczorek, M., \&
Frazier, N. (2020). Mainstems: A logical data model implementing
mainstem and drainage basin feature types based on WaterML2 part 3: HY
features concepts. \emph{Environmental Modelling \& Software},
\emph{135, 104927}.

\leavevmode\vadjust pre{\hypertarget{ref-8}{}}%
Clarke, K., \& \textbf{Johnson, J.} (2020). Calibrating SLEUTH with big
data: Projecting california's land use to 2100. \emph{Computers,
Environment and Urban Systems}, \emph{83, 101525}.

\leavevmode\vadjust pre{\hypertarget{ref-9}{}}%
Wens, M., Veldkamp, T., Mwangi, M., \textbf{Johnson, J.}, Lasage, R.,
Haer, T., \& \ldots. (2020). Simulating small-scale agricultural
adaptation decisions in response to drought risk: An empirical
agent-based model for semi-arid kenya. \emph{Frontiers in Water},
\emph{2, 15}.

\leavevmode\vadjust pre{\hypertarget{ref-11}{}}%
Clarke, K., \textbf{Johnson, J.}, \& Trainor, T. (2019). Contemporary
american cartographic research: A review and prospective.
\emph{Cartography and Geographic Information Science}, \emph{46 (3),
196-209}.

\leavevmode\vadjust pre{\hypertarget{ref-13}{}}%
\textbf{Johnson, J.}, Munasinghe, D., Eyelade, D., \& Cohen, S. (2019).
An integrated evaluation of the national water model (NWM) height above
nearest drainage (HAND) flood mapping methodology. \emph{Natural Hazards
and Earth System Sciences (NHESS)}.

\leavevmode\vadjust pre{\hypertarget{ref-12}{}}%
\textbf{Johnson, J.}, Wens, M., Zagaria, C., \& Veldkamp, T. (2019).
Integrating human behavior dynamics into drought risk assessment---a
sociohydrologic, agent‐based approach. \emph{Wiley Interdisciplinary
Reviews: Water, e}, \emph{e1345}.

\leavevmode\vadjust pre{\hypertarget{ref-16}{}}%
Blodgett, D., \& \textbf{Johnson, J.} (2018). nhdplusTools: Tools for
accessing and working with the NHDPlus. \emph{Avaiable from
Https://Code. Usgs. Gov/Water/nhdplusTools}.

\leavevmode\vadjust pre{\hypertarget{ref-17}{}}%
De Cicco, L., Lorenz, D., Hirsch, R., Watkins, W., \&
\textbf{Johnson, J.} (2018). dataRetrieval: R packages for discovering
and retrieving water data available from US federal hydrologic web
services. \emph{US Geological Survey, Reston, VA, Https://Doi. Org/},
\emph{/10.5066/P9X4L3GE}.

\leavevmode\vadjust pre{\hypertarget{ref-15}{}}%
\textbf{Johnson, J.}, Coll, J., Ruess, P., \& Hastings, J. (2018).
Challenges and opportunities for creating intelligent hazard alerts: The
{``FloodHippo''} prototype. \emph{JAWRA Journal of the American Water
Resources Association}.

\leavevmode\vadjust pre{\hypertarget{ref-14}{}}%
Lo'aiciga, H., \& \textbf{Johnson, J.} (2018). Infiltration on sloping
terrain and its role on runoff generation and slope stability.
\emph{Journal of Hydrology}, \emph{561, 584-597}.

\leavevmode\vadjust pre{\hypertarget{ref-18}{}}%
\textbf{Johnson, J.}, Coll, J., Cohen, S., Nelson, J., Ogden, F.,
Praskievicz, S., \& \ldots. (2017). National water center innovators
program summer institute report 2017. \emph{Consortium of Universities
for the Advancement of Hydrologic Science, Inc.}

\leavevmode\vadjust pre{\hypertarget{ref-19}{}}%
\textbf{Johnson, J.}, \& Lo'aiciga, H. (2017). Coupled infiltration and
kinematic-wave runoff simulation in slopes: Implications for slope
stability. \emph{Water}, \emph{9 (5), 327}.

\hypertarget{grants-and-fellowships}{%
\section{Grants and Fellowships}\label{grants-and-fellowships}}

\textcolor{cyan}{I have personally solicited \textbf{\$451,000} for research and development and been a core member of teams who have solicited \textbf{\$19,292,519}.}

\begin{cventries}
    \cventry{NOAA Office of Water Prediction}{NOAA OWP Geospatial Services}{\$8,000,000}{2023-2025}{}\vspace{-4.0mm}
    \cventry{NOAA Office of Water Prediction}{NOAA OWP Next Generation Water Resource Modeling Framework Development}{\$7,300,000}{2022-2024}{}\vspace{-4.0mm}
    \cventry{Earth Science Information Partners}{Increasing Environmental Data Access through a more robust federated data catalog and extending the climateR model to Python}{\$6,000}{2023}{}\vspace{-4.0mm}
    \cventry{Earth Science Information Partners}{Machine Learning for Flood Risk Assessment}{\$20,000}{2022}{}\vspace{-4.0mm}
    \cventry{National Science Foundation}{The UFOKN: Delivering Flood Information to AnyOne, AnyTime, AnyWhere}{\$2,853,561 (Subaward: \$240,000)}{2020-2022}{}\vspace{-4.0mm}
    \cventry{National Science Foundation}{Convergence Accelerator Phase I (RAISE): The Urban Flooding Open Knowledge Network (UFOKN)}{\$1,027,958 (Subaward: \$100,000)}{2019-2020}{}\vspace{-4.0mm}
    \cventry{UCAR COMET}{A National Water Model R Package: Improving access and application of model output}{\$15,000}{2018-2019}{}\vspace{-4.0mm}
    \cventry{UCAR COMET}{FOSSFlood: The LivingFlood Application Built on Free Open Source Software}{\$5,000}{2017-2018}{}\vspace{-4.0mm}
    \cventry{UCGHI Planetary Health Seed Grant}{Integrating farmers’ adaptive behaviors in California’s Central Valley to assess water and food security risks under climate change}{\$10,000}{2017-2018}{}\vspace{-4.0mm}
    \cventry{CUAHSI}{CUAHSI HydroInformatics Fellowship}{\$5,000}{2020-2021}{}\vspace{-4.0mm}
    \cventry{Jack and Laura Dangermond}{Jack and Laura Dangermond GIS Fellow in Residence}{\$5,000}{2019-2020}{}\vspace{-4.0mm}
    \cventry{CUAHSI}{National Water Center Summer Institute}{\$15,000}{2016}{}\vspace{-4.0mm}
    \cventry{University of California Regents}{Disciplines Fellowship}{\$30,000}{2015-2016}{}\vspace{-4.0mm}
\end{cventries}

\hypertarget{teaching-experience}{%
\section{Teaching experience}\label{teaching-experience}}

\textcolor{cyan}{I designed an upper division spatial data science course as a UCSB Lecturer, was a teaching assistant for over 15 courses (700+ students), and have lead community workshops for national organizations.}

\hypertarget{section}{%
\subsection{\texorpdfstring{\textcolor{blue}{University Teaching}}{}}\label{section}}

\begin{cventries}
    \cventry{University of California, Santa Barbara, California}{Introduction to Geoinformatics}{Santa Barbara, CA}{2021}{\begin{cvitems}
\item Independently developed and taught to address the growing need for data science in the GIS profession.
\item Intended to become prerequisite course for the UCSB Geography Department and Masters in GIS Curriculum
\item \href{https://mikejohnson51.github.io/spds/}{Open course content available here}
\end{cvitems}}
\end{cventries}

\hypertarget{section-1}{%
\subsection{\texorpdfstring{\textcolor{blue}{Teaching Assistant}}{}}\label{section-1}}

\begin{cventries}
    \cventry{Dr. Vena Chu, Alana Ayasse}{Remote Sensing of the Environment 2}{2021, 2020}{Upper-Division}{}\vspace{-4.0mm}
    \cventry{Dr. Catherine Gautier}{Living with Global Warming}{2020, 2019, 2018, 2016}{Lower-Division}{}\vspace{-4.0mm}
    \cventry{Dr. Krzysztof Janowicz}{Conceptual Modeling and Programming for the Geo-Sciences}{2020, 2019, 2017}{Upper-Division and Graduate}{}\vspace{-4.0mm}
    \cventry{Dr. Joe McFadden}{Remote Sensing of the Environment 1}{2020}{Upper-Division}{}\vspace{-4.0mm}
    \cventry{Dr. Vena Chu}{Remote Sensing of the Environment 3}{2019}{Upper-Division}{}\vspace{-4.0mm}
    \cventry{Dr. Werner Kuhn, Dr. Keith Clarke}{Maps and Spatial Reasoning}{2019, 2018, 2017}{Lower-Division}{}\vspace{-4.0mm}
    \cventry{Dr. Keith Clarke}{Cartographic Design and Geovisualization}{2018}{Upper-Division}{}\vspace{-4.0mm}
    \cventry{Dr. Hugo Loaiciga}{Environmental Water Quality}{2017}{Upper-Division}{}\vspace{-4.0mm}
    \cventry{Dr. Tim DeVeries}{Oceans and Atmosphere}{2016}{Lower-Division}{}\vspace{-4.0mm}
\end{cventries}

\hypertarget{section-2}{%
\subsection{\texorpdfstring{\textcolor{blue}{Workshops}}{}}\label{section-2}}

\begin{cventries}
    \cventry{NOAA 2023 Summer Institute}{Leveraging the NHGF and NextGen derived products for Research}{June 2023}{Workshop Lead}{}\vspace{-4.0mm}
    \cventry{CIROH Training and Developer’s conference}{The NextGen Hydrofabric: What Is It, How to get it, and how to make your own?}{May 2023}{Workshop Lead}{\begin{cvitems}
\item Design and led 2 workshops exposing over 100 new developers to the avaialbe tools, data models, and dataset developed.
\end{cvitems}}
    \cventry{NOAA 2022 Summer Institute}{Introduction to core hydrofabric services and concepts}{June 2022}{Workshop Lead}{}\vspace{-4.0mm}
    \cventry{Internet of Water}{Working with Geospatial Hydrologic Data Using Web Services}{July 2022}{Workshop Co-lead}{}\vspace{-4.0mm}
    \cventry{AWRA 2022 Geospatial Water Technology Conference}{R and Python Tools for Geospatial Water Applications}{May 2022}{Workshop Co-lead}{}\vspace{-4.0mm}
\end{cventries}

\hypertarget{section-3}{%
\subsection{\texorpdfstring{\textcolor{blue}{Award Nominations}}{}}\label{section-3}}

\begin{cventries}
    \cventry{}{Nominated for UCSB GSA Excellence in Teaching by students}{}{2020, 2019}{}\vspace{-4.0mm}
    \cventry{}{Nominated for UCSB Geography Excellence in Teaching by faculty member}{}{2020, 2019}{}\vspace{-4.0mm}
\end{cventries}

\hypertarget{open-source-software}{%
\section{Open Source Software}\label{open-source-software}}

\textcolor{cyan}{A primary output of my scientific work is open source software in personal, USGS and NOAA repositories.}
\newline \newline
\textcolor{cyan}{\href{https://github.com/mikejohnson51}{\faicon{github} Github: \faicon{user} 179 followers; \faicon{star} 539 stars}}

\begin{cventries}
    \cventry{Fast and flexible geocoding and AOI creation.}{\href{https://github.com/mikejohnson51/AOI}{AOI}}{}{Lead Developer}{}\vspace{-4.0mm}
    \cventry{Instant access to gridded and observation climate data.}{\href{https://github.com/mikejohnson51/climateR}{climateR}}{}{Lead developer}{}\vspace{-4.0mm}
    \cventry{A consistent federated data catalog for programmatic access.}{\href{https://github.com/mikejohnson51/climateR-catalogs}{climateR-catalogs}}{}{Lead developer}{}\vspace{-4.0mm}
    \cventry{Fast, flexable spatial data summarization.}{\href{https://github.com/mikejohnson51/zonal}{zonal}}{}{Lead developer}{}\vspace{-4.0mm}
    \cventry{National Water Model Streamflow access.}{\href{https://github.com/mikejohnson51/nwmTools}{nwmTools}}{}{Lead developer}{}\vspace{-4.0mm}
    \cventry{Manipulating hydrographic data with the NHDPlus data model.}{\href{https://github.com/doi-usgs/nhdplusTools}{DOI-USGS/nhdplusTools}}{}{Author}{}\vspace{-4.0mm}
    \cventry{R Interface to the USGS data holdings.}{\href{https://github.com/doi-usgs/dataRetrieval}{DOI-USGS/dataRetrieval}}{}{Author}{}\vspace{-4.0mm}
    \cventry{Manipulating the NHDPlus Network for Hydrologic Modeling.}{\href{https://github.com/doi-usg/hyRefactor}{DOI-USGS/hyRefactor}}{}{Author}{}\vspace{-4.0mm}
    \cventry{Generating data products for continental scale hydrology}{\href{https://github.com/NOAA-OWP/hydrofabricI}{NOAA-OWP/hydrofabric}}{}{Lead Developer}{}\vspace{-4.0mm}
    \cventry{Estimating robust, mass conserving AHG relationships with cross section hydrualics and geometry}{\href{https://github.com/mikejohnson51/FHGestimation}{FHGestimation}}{}{Lead Developer}{}\vspace{-4.0mm}
\end{cventries}

\hypertarget{invited-presentations}{%
\section{Invited Presentations}\label{invited-presentations}}

\begin{cventries}
    \cventry{AGU San Fransisco}{Current State of the NOAA NextGen Enterprise Hydrofabric System}{Dec 2023 (Tenative)}{Conference Talk}{}\vspace{-4.0mm}
    \cventry{ICF Global Headquarters Conference Center}{Integrated Hydro-Terrestrial Modeling 2.0}{Oct 2023}{Workshop}{\begin{cvitems}
\item Workshops to advance community modeling and integrated water resources management.
\item  Nominated by NOAA to attend.
\end{cvitems}}
    \cventry{ESIP Rants and Raves: Information Technology and Interoperability (IT\&I) Tech Dive}{Meeting Data Where it Lives the power of virtual access patterns}{Mar 2023}{Tech Talk}{\begin{cvitems}
\item Exploring the underutilized potetnial of GDAL virtual access patterns in a 1 hour technical talk.
\end{cvitems}}
    \cventry{AGU Chicago}{The NOAA NextGen Water Resources Modeling Framework Hydrofabric: Version 1.0}{Dec 2022}{Conference Talk}{}\vspace{-4.0mm}
    \cventry{AGU Chicago}{Introducing a building level, continental scale, flood risk forecast system}{Dec 2022}{Conference Talk}{}\vspace{-4.0mm}
    \cventry{NOAA-USGS Quarterly Meetings}{NOAA-USGS Quarterly Meetings}{Nov 2022}{Tech Talk}{\begin{cvitems}
\item Briefed USGS and NOAA Leadership at Quartly Meeting. 
\item  Represented ongoing NOAA USGS collaboration.
\end{cvitems}}
    \cventry{National Conservation Training Center Facility}{NOAA USGS Modeling Workshop}{Oct 2022}{Stratigic Planning Workshop}{\begin{cvitems}
\item USGS/NOAA Programatic Level Setting
\end{cvitems}}
    \cventry{Fronteirs in Hydrology: Puerto Rico}{End-to-end Hydrofabric workflows for the NextGen Water Resources Modeling Framework}{Jun 2022}{Conference Talk}{}\vspace{-4.0mm}
    \cventry{AGU: New Orleans}{Tools for Processing the NHDPlus into a Hydrofabric Suitable for Use in the NextGen National Water Model}{Dec 2021}{Conference Talk}{}\vspace{-4.0mm}
\end{cventries}



\end{document}
